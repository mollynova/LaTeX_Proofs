\documentclass[12pt]{article}
\usepackage{amsmath}
\usepackage{amssymb}
\usepackage{amsthm}
\usepackage{latexsym}
\usepackage{graphicx}
\usepackage{bm}
\usepackage{indentfirst}
\usepackage{booktabs} % for "\midrule" macro


\author{Molly Novash}
\title{%
\Huge Proof the Eighth for CS250  \\
\normalsize Powered by \LaTeXe}

\begin{document}
\sloppy
\maketitle
\bigskip

This document will prove the following statement, by induction \ldots \\

\begin{equation}
\sum_{i = 1}^{n} i^{3} = \left(\frac{(n(n + 1))}{2}\right)^2
\end{equation}
\bigskip

\begin{proof}
We begin our induction proof by identifying $P_{n}$ \ldots \\

\begin{equation}
P_{n} = \sum_{i = 1}^{n} i^{3} = \left(\frac{(n(n + 1))}{2}\right)^2
\end{equation}
\bigskip

The sum of the values of $i \in \mathbb{N}$ from 1 to $n$, raised to the third power, is equal to $n$, times $n + 1$, divided by two, all squared. \\

Next, we illustrate the truth of $P(1)$, where $n = 1$ is true. \\

\begin{equation}
P(1) = \sum_{i = 1}^{1} i^{3} = 1^3 = 1
\end{equation}

\begin{equation}
\equiv P(1) = \left(\frac{(1(1+1))}{2}\right)^2
\end{equation}

\begin{equation}
P(1) = \left(\frac{2}{2}\right)^2 = 1^2 = 1
\end{equation}
\bigskip

The value of the summation for $P(1)$ and the value of the expression we wish to prove is equivalent are both 1 for $n = 1$. \\

In the inductive step, we shall assume that $P(k)$ is true, for a particular but arbitrarily chosen $k$ $\in$ $\mathbb{N}$. \\

$P(k)$ is obtained by substituting $k$ for every $n$. This is our inductive hypothesis. \ldots \\

\begin{equation}
P_{k} = \sum_{i = 1}^{k} i^{3} = \left(\frac{(k(k + 1))}{2}\right)^2 = 1^3 + 2^3 + ... + (k - 1)^3 + k^3
\end{equation}
\bigskip

Similarly, $P(k + 1)$ is obtained by substituting the quantity $(k + 1)$ for every $n$ in $P(n)$ \ldots \\

\begin{equation}
P_{k + 1} = \sum_{i = 1}^{k + 1} i^{3} = \left(\frac{(k + 1)(k + 2))}{2}\right)^2 = 1^3 + 2^3 + ... + k^3 + (k + 1)^3
\end{equation}
\bigskip

Notice that the extended summation of $P(k)$ is identical to that of $P(k + 1)$, save for the addition of the final expression, $(k + 1)^3$. By algebra, we can therefore substitute the expression which is equal to $P(k)$ into the extended summation of $P(k + 1)$, which accounts for all values except for $(k + 1)^3$. \\

Solving algebraically, if the resulting expression is equivalent to the expression representing the summation of $P(k + 1)$, we will have proven that this expression holds as an accurate equivalence of the summation in all cases. \\

By substitution, therefore, from our inductive hypothesis \ldots \\

\begin{equation}
P_{k + 1} = \sum_{i = 1}^{k + 1} i^{3} = \left(\frac{(k(k + 1))}{2}\right)^2 + (k + 1)^3
\end{equation}
\bigskip

By "foiling" the two expressions, we obtain the following \ldots \\

\begin{equation}
P_{k + 1} = \sum_{i = 1}^{k + 1} i^{3} = \left(\frac{k^4 + 2k^3 + k^2}{4}\right) + (k^3 + 3k^2 + 3k + 1)
\end{equation}
\bigskip

Multiplying the second term by $\frac{4}{4}$, we obtain a common denominator \ldots \\

\begin{equation}
P_{k + 1} = \sum_{i = 1}^{k + 1} i^{3} = \left(\frac{k^4 + 2k^3 + k^2}{4}\right) + \left(\frac{4(k^3 + 3k^2 + 3k + 1)}{4}\right)
\end{equation}
\bigskip

By adding fractions with a common denominator \ldots \\

\begin{equation}
= \sum_{i = 1}^{k + 1} i^{3} = \left(\frac{k^4 + 6k^3 + 13k^2 + 12k + 4}{4}\right)
\end{equation}
\bigskip

Finally, we will "foil" out our original $P(k + 1)$ expression, to see that they're the same  \ldots \\

\begin{equation}
P_{k + 1} = \sum_{i = 1}^{k + 1} i^{3} = \left(\frac{(k + 1)(k + 2)}{2}\right)^2 
\end{equation}
\bigskip

\begin{equation}
= \sum_{i = 1}^{k + 1} i^{3} = \left(\frac{(k^2 + 3k + 2)}{2}\right)^2 
\end{equation}
\bigskip

\begin{equation}
= \sum_{i = 1}^{k + 1} i^{3} = \left(\frac{(k^2 + 3k^3 + 2k^2 + 3k^3 + 9k^2 + 6k + 2k^2 + 6k + 4}{4}\right)
\end{equation}
\bigskip

\begin{equation}
= \sum_{i = 1}^{k + 1} i^{3} = \left(\frac{k^4 + 6k^3 + 13k^2 + 12k + 4}{4}\right)
\end{equation}
\bigskip

As has clearly been demonstrated, expressions (11) and (16) are identical. Therefore, the original expression has been proven to be an accurate representation of our summation for all $n$ $\in$ $\mathbb{N}$.

\end{proof}
\end{document}